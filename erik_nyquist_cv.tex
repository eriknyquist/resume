\documentclass[6pt]{article}

\usepackage[none]{hyphenat}
\usepackage{titlesec}
\usepackage{enumitem}

\titlespacing*{\section}{0pt}{0.0\baselineskip}{\baselineskip}
\addtolength{\topmargin}{-10em}
\addtolength{\oddsidemargin}{-8em}
\addtolength{\evensidemargin}{-8em}

\titleformat*{\section}{\tiny}
\textheight=65em

\titlespacing{\section}{0pt}{2ex}{1ex}
\titlespacing{\subsection}{0pt}{1ex}{0ex}
\titlespacing{\subsubsection}{0pt}{0.5ex}{0ex}

\setlist[itemize]{leftmargin=*}

\begin{document}
%
% Start of "Personal Details" section (sidebar)
%
\hspace*{-\parindent}%
\begin{minipage}{15em}
\section*{}
{\underline{\textbf{Personal Details}}}
\break

{\small Website: https://www.ekn.io

Github: eriknyquist

Email: eknyquist@gmail.com}

\section*{}
{\underline{\textbf{Areas of Expertise}}}
\subsection*{}
{\textbf{Programming Languages}}

{\small C

C++

Python

UNIX shell scripting (bash, sh)}

\subsection*{}
{\textbf{Tools \& Environments}}

{\small Git

GCC/Clang

Makefiles

GNU ld (linker) scripting

Google Protocol Buffers

GDB

OpenOCD

Valgrind

LaTeX

Eclipse

Jenkins

JIRA}

\subsection*{}
{\textbf{Personal Skills}}

\noindent
\begin{itemize}
{\small \raggedright
    \item Test-driven development
    \item Hardware-level fault finding and debugging
    \item C and C++ for memory constrained embedded systems
}
\end{itemize}

\subsection*{}
{\textbf{Interests}}

\noindent
\begin{itemize}
{\small \raggedright
    \item Compiler design and implementation
    \item Programming language design and implementation
    \item Playing music (drums, piano)
}
\end{itemize}
\end{minipage}
%
% Start of main section (page 1)
%
\begin{minipage}{35em}
%
% Document title
%
{\Huge \bfseries Erik Nyquist}
\section*{}

An enthusiastic and skillful software engineer, with a comprehensive knowledge of
development and validation practices for embedded software systems. Accustomed to
delivering and enforcing high quality code and documentation in a high-pressure
environment.

\section*{}
{\Large \bfseries Experience}
%
% Current job
%
\subsection*{}
\hspace*{-\parindent}%
\begin{minipage}{20em}
{\bfseries Firmware Engineer, NOVO Engineering \\
Vista, CA}
\end{minipage}
\hfill
\begin{minipage}{10em}
{
    \bfseries \hfill Aug. 2017 - present \\

}
\end{minipage}
\break
\break
Designing and developing low-level firmware and user-level software for embedded
systems (Linux, FreeRTOS)
\begin{itemize}
    \item Developed firmware for multiple pieces of product test equipment on nRF SoCs
    \item Designed an interface and developed a BLE peripheral application (Python) for an
          embedded Linux-based system, to enable communication with the iOS and Android
          apps used to configure and monitor the system.
    \item Developed USB drivers (C, C++) for a FreeRTOS-based product prototype system
    \item Developed automated unit tests and integration tests (Python) for several products
\end{itemize}
\hspace*{-\parindent}%
\begin{minipage}{20em}
%
% Previous job
%
\subsection*{}
{\bfseries Software Engineer, Intel \\
San Diego, CA}
\end{minipage}
\hfill
\begin{minipage}{12em}
{
    \bfseries \hfill Aug. 2016 - Aug. 2017 \\

}
\end{minipage}
\break
\break
Developed low-level hardware drivers and firmware for Intel's low-power SoC products
with a small team, including Intel's Galileo, Joule and Curie modules (Linux, RTOS and
bare-metal).
\begin{itemize}
    \item Developed drivers and firmware (C, C++) for the Intel Arduino 101 board, using the
          open-source Zephyr project (co-operative scheduling real-time kernel for small
          embedded systems. www.github.com/01org/zephyr).
    \item Developed Arduino user libraries and APIs (C, C++) for the Intel Arduino 101 board,
          reviewed/accepted patches from public users regularly \\
          (www.github.com/01org/corelibs-arduino101).
    \item Developed drivers and developer tutorials (C++) for the Pattern Matching Accelerator
          on the Quark SE C1000 microcontroller (on-die network of 128 hardware neurons
          capable of autonomous learning/training, and pattern recognition.
          www.github.com/01org/Intel-Pattern-Matching-Technology).
    \item Participated in design and development of the Curie Open Developer Kit (C, C++,
          bash/shell, OpenOCD, GDB), a lightweight framework for developing applications
          (bare-metal, or using the Zephyr kernel) that can fully utilize the two asynchronous
          CPU cores on an Intel Curie device (www.intel.com/curieodk).
\end{itemize}
\end{minipage}
%
% Start of main section (page 2)
%
\hspace{2em}
\begin{minipage}{50em}
\begin{minipage}{20em}
%
% Previous job
%
\subsection*{}
{\bfseries SoC Software Engineer, Intel

San Diego, CA}

\end{minipage}
\hfill
\begin{minipage}{28em}
{
    \bfseries \hfill Sep. 2014 - Apr. 2016

}
\end{minipage}
\break
\break
Developed and tested low-level hardware drivers and firmware for Quark SoCs (Linux and bare-
metal).
\begin{itemize}
    \item Developed Linux-based test suites (C, Python), for various embedded protocols and
          systems (UART, CAN, I2C, SPI), on FPGA-emulated systems and during new device
          bring-up/power-on.
    \item Developed test suites (C, Python) for customer SW tools (tools for writing BIOS
          configuration data to an EEPROM/flash device on a customer board, tools for
          generating a bootable OS image with custom BIOS, kernel and root filesystem parts)
    \item Functional testing of customer SW tools \& platforms (microcontroller I/O libraries, toolchain,
          debugging tools).
    \item Low-level debugging and fault-finding (source and assembly level debug with OpenOCD and GDB,
          analyzing linker scripts, linker map files, and looking at scope traces / logic analyzer traces, to
          debug I/O device drivers or C/C++ run-time init. code running on a target microcontroller device)
\end{itemize}

\hspace*{-\parindent}%
\begin{minipage}{20em}
%
% Previous job
%
\subsection*{}
{\bfseries Software Design Engineer, Intel

Ireland}

\end{minipage}
\hfill
\begin{minipage}{14em}
{
    \bfseries \hfill Aug. 2012 - Sept. 2014

}
\end{minipage}
\break
\break
Developed and tested Linux-based software \& drivers for the Intel Quark X1000 SoC. Brought Intel
Galileo board (first x86-based Arduino board) from design to market.
\begin{itemize}
    \item Prototyped software solutions for supporting Arduino-compatible code on an embedded linux-
          based platform
    \item Usability and functional testing of customer SW tools (Code editor/IDE, setup tools and scripts)
    \item Provided technical support to users via online forum www.maker.intel.com
\end{itemize}

\section*{}
{\Large \bfseries Education}
\subsection*{}
{\bfseries Master of Science, Computer Science}

University College Dublin, Belfield, Ireland
\subsection*{}
{\bfseries Bachelor of Engineering, Audio Visual Media Technology}

Dun Laoghaire Institute of Art, Design and Technology, Dublin, Ireland

\end{minipage}
\end{document}
